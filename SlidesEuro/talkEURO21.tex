% \documentclass[notes]{beamer}       % print frame + notes
\documentclass{beamer}   % only notes
%\documentclass{beamer}              % only frames
\usepackage{LlabsTheme}

\usepackage[utf8]{inputenc}
\usepackage[T1]{fontenc}
\usepackage[german, english]{babel}
\usepackage{amsmath,amssymb,amsthm}
\usepackage{makeidx}
\usepackage{graphicx}
\usepackage{xspace}
\usepackage{url}
\usepackage{comment}
\usepackage{caption}

\usepackage{memhfixc}
\usepackage{hyphenat}
\usepackage{xcolor}

\usepackage[square,numbers]{natbib}
\bibliographystyle{abbrvnat}

\usepackage{afterpage}
\usepackage{refcount}
\usepackage{graphicx}

\usepackage{tikz}

\usepackage{url}
\usepackage{comment}
\usepackage{xspace}
\usepackage{siunitx}
\sisetup{locale = US}
\DeclareSIUnit{\mph}{mph}
\setlength\abovecaptionskip{2pt}

\captionsetup[figure]{font=tiny,labelfont=tiny}

%% to propeerly type 1st 2nd 3rd, usw
\newcommand\nd{\textsuperscript{nd}\xspace}
\newcommand\rd{\textsuperscript{rd}\xspace}
\newcommand\st{\textsuperscript{st}\xspace}

\newcommand\nth{\textsuperscript{th}\xspace} %\th is taken already


%\usepackage{algorithm}
%\usepackage{algpseudocode}

% \usepackage{algorithm2e}
% \usepackage{algorithmic}

%\usepackage{ngerman}      % language set to new-german
\usepackage[utf8]{inputenc}   % coding of german special characters
\usepackage[absolute,overlay]{textpos}
\usepackage{array}
\usepackage{setspace}
\newcommand{\RNum}[1]{\uppercase\expandafter{\romannumeral #1\relax}}
\newcommand{\argmin}{\operatornamewithlimits{argmin}}
\newcommand{\argmax}{\operatornamewithlimits{argmax}}
\newcommand{\meth}[1]{\texttt{\textbf{#1}}}

\newcommand{\csharp}{C\#\xspace}

\newcommand{\HL}[1]{{\color{primarycolor} #1}}

\usepackage{tikz}

\usepackage{hyperref}
\usepackage{graphicx}

\newcommand{\coo}{\ensuremath{\mathrm{CO_2}}}

\newcounter{thmcount}
\newtheoremstyle{break}% name
{}%         Space above, empty = `usual value'
{}%         Space below
{}% Body font
{}%         Indent amount (empty = no indent, \parindent = para indent)
{\bfseries}% Thm head font
{}%        Punctuation after thm head
{\newline}% Space after thm head: \newline = linebreak
{}%         Thm head spec

\usepackage{etoolbox}


\theoremstyle{break}
\newtheorem{exmp}{Example}[thmcount]
\newtheorem{obs}{Observation}[thmcount]
\newtheorem{cor}{Corollary}[thmcount]
\newtheorem{defn}{Definition}[thmcount]
\newtheorem{lem}{Lemma}[thmcount]

\usepackage{adjustbox}

\newcommand{\outgoing}{\emph{outgoing}\xspace}
\newcommand{\return}{\emph{return}\xspace}

\newcommand{\punctual}{\emph{punctual}\xspace}
\newcommand{\tardy}{\emph{tardy}\xspace}

\renewcommand*{\thefootnote}{\fnsymbol{footnote}}



\title[Route selection for oversize and heavyweight
cargo transportation]{
\large
Optimal route selection for oversize and heavyweight
cargo transportation considering bridge carrying capacity
}
% \thanks{This work is supported by Lakeside Labs GmbH, Klagenfurt, Austria and funding from
% the European Regional Development Fund and the Carinthian Economic Promotion Fund (KWF) under grant 20214/31942/45906.}
% }

\author[Christian Wankm\"uller]{\small
Christian Wankm\"uller  \inst{1}
\and
Andreas Felsberger \inst{1}
\and
Christian Truden \inst{2}
}

\institute[LLabs]{\footnotesize
\inst{1} Department of Operations Management and Logistics, Alpen-Adria-Universit\"at Klagenfurt,
Klagenfurt, Austria
\inst{2} Lakeside Labs GmbH, Klagenfurt, Austria
}

\date{\footnotesize
31\st European Conference on Operational Research,\\
University of West Attica, Athens, Greece,\\
July 11-16, 2021
}





%Customized settings to define the style of the presentation

%%%%COLORS %%%%
%defines the used color scheme
\colorsAAU

%\titlelogoAAU  %%AAU logo on title frame
% \titlelogoLLabsLUC  %%LLAbs and LUC logo on title frame
% \titlelogoLLabsEFRE
\titlelogoAAULLabs

%%%%LOGOS ON FRAMES %%%
%defines which logos are displayed on the frame, if you want to change the logos
%for certain frames, just apply the command before the \begin{frame} command
%all following frames will have the defines appearance
% \logoAAU
% \logoLlabsLUC
\logoAAULlabs
%\logoNONE

%%%HEADLINE
\headlineLEFT  %aligns the section overview in the headline LEFT
%\headlineCENTER %spreads the  section overview in the headline evenly


%%%FOOTLINE
%Define the appearance of the footline
%\footlineNUMBER  % show number of current slide
\footlineNUMBERTOTAL  % also shows the total number of slides
%\footlineNONUMBER  %no number displays

%%%%AVAILABLE COLORS%%%
%the following colors are predefined for use with the \color{col} command
%-   primarycolor
%-  secondarycolor

\begin{document}


\begin{frame}[plain]{\titlepage}\end{frame}
%
%
%
%
%
%   \section{Introduction}
%
%
  \begin{frame}
\frametitle{Intro}
\begin{itemize}
  \item
\end{itemize}
\end{frame}
%     \frametitle{Motivation - Passenger Transportation in Rural Areas}
%
%     \begin{itemize}
%       \item Transportation demand arises from the need to reach urban centers (work, school, etc.).
%       \item Demand peaks around particular times due to low population density.
%       \item Lack of transport provision for the first/last mile to/from public transport system corridors (timetabled services).
%       \item Need for sustainable and reliable form of rural passenger mobility.
%       \item Individual car use is (often) the only means of transportation.
%       \item Major source of \coo-emissions (besides industry).
%     \end{itemize}
%   \end{frame}
%
%
%   \begin{frame}
%     \frametitle{Micro-Transit Systems}
%     \begin{itemize}
%       \item Demand-responsive ride-sharing options that are flexible in their service provision.
%       \item Service provider's perspective
%       \begin{itemize}
%         \item \HL{Usage behavior} of passengers that use the system.
%         \item \HL{Service provision}, service may vary in terms of modality, purpose, flexibility.
%       \end{itemize}
%       \item Customers may request transports
%       \begin{itemize}
%         \item \HL{Pre-booked}, well ahead of journey start time.
%         \item \HL{Ad-hoc}, close to actual required travel time.
%       \end{itemize}
%       \item A successful service must find a balance between
%       \begin{itemize}
%         \item Convenience: all transport requests (including delayed) are serviced.
%         \item Reliability: destinations are reached in time.
%       \end{itemize}
%
%     \end{itemize}
%   \end{frame}
%
%
%
%
%   \section{Approach}
%
%
%
%
%   \begin{frame}
%     \frametitle{Transport Demand}
%     \begin{itemize}
%       \item \HL{Characteristics of a Transport Request}
%       \begin{itemize}
%         \item \textit{Origin / Destination}.
%         \item \textit{Arrival Time Window} for transferring to time-tabled services.
%         \item \textit{Pick-Up Time} at which passenger must be ready at location (given by Optimization).
%       \end{itemize}
%       \item
%       \HL{Transport Request Chain}
%       \begin{itemize}
%         \item Several transport requests, which together define a sequence of journey legs, are scheduled in an "all or nothing" fashion by a provider.
%         \item A transport chain is regarded as ``broken'' when a passenger misses the pick-up time, and ceases to be serviced.
%       \end{itemize}
%
%       \item Commuter requests
%       \begin{itemize}
%         \item \textit{Outgoing request}: towards train/bus stations in the morning.
%         \item \textit{Return request}: in the evening.
%       \end{itemize}
%
%     \end{itemize}
%   \end{frame}
%
%
%
%   \begin{frame}
%     \frametitle{Challenges for the Service Provider}
%
%     \begin{itemize}
%       \item How does the \HL{punctuality} of the passengers affect service provision?
%       \vspace*{0.4cm}
%       \item Should drivers wait for late passengers? For how long?
%       \vspace*{0.4cm}
%       \item How can the service provider identify a reasonable policy for how to deal with late (unpunctual) passengers?
%     \end{itemize}
%   \end{frame}
%
%
%   \begin{frame}
%     \frametitle{Overall Approach}
%     \vspace*{-1cm}
%     \begin{adjustbox}{max totalsize={.9\textwidth}{.7\textheight},center}
%       \begin{tikzpicture}
  \node (opt) at (0,0) [draw,thick,minimum height=0.8cm] {\bf Optimization};
  \node (sim) at (5,0) [draw,thick,minimum height=0.8cm] {\bf Simulation};
  \node (pop) at (-2.5, 1.2)  {\begin{tabular}{c}Synthetic Population \\ (OD-Pairs) \end{tabular} };
  \node (res) at (7, 1.2)  {\begin{tabular}{c}Experimental \\ Results \end{tabular} };
  \draw[->,line width=0.8mm] (opt) -- (sim)  node [midway,align=center]{\begin{tabular}{c}Transport \\ Schedules \end{tabular}}  ;
  \draw[->,line width=0.8mm]  (pop) -- (-2.5,0)  -- (opt);
  \draw[->,line width=0.8mm]  (sim) -- (7,0)  -- (res);
\end{tikzpicture}

%     \end{adjustbox}
%     \begin{enumerate}
%       \item Create transport schedules using \textit{Large Neighborhood Search} [Ropke \& Pisinger 2006] for
%       a multi-objective variant of the \textit{Dial-a-ride problem} (DARP).
%       \item Use Agent-based Modeling (ABM) and Simulation to analyze the influence of external factors on the execution of transport schedules.
%       \item Fine-tune parameters of service provision based on the simulation results.
%     \end{enumerate}
%   \end{frame}
%
%
%   \begin{frame}
%     \frametitle{Agent-based Model}
%
%     \begin{itemize}
%       \item \HL{Cycle.} \textit{Event} (time) $\rightarrow$ \textit{Activity} (randomness) $\rightarrow$ \textit{Event}.
%       \item \HL{Vehicle Agents}
%       \begin{itemize}
%         \item Encapsulate vehicles and assigned drivers.
%         \item Mission: follow the transport schedule.
%         \item Activities: (i) transfer between stops, (ii) arrive (leave) at (from) stop, (iii) wait for passengers, etc.
%       \end{itemize}
%       \item \HL{Passenger Agents}
%       \begin{itemize}
%       \item Mission: complete their trip plan (derived from vehicle schedules).
%       \item Activities: (i) walk to location, (ii) arrive at location,
%       (iii) (un)board vehicle, (iv) wait for vehicle, etc.
%       \end{itemize}
%       \item \HL{Stop Agents}
%       \begin{itemize}
%         \item Passive role, serve as ``interface'' between passengers and vehicles.
%       \end{itemize}
%     \end{itemize}
%   \end{frame}
%
%
%   \section{Analysis}
%
%   \begin{frame}
%     \frametitle{Passenger Types}
%
%     \vspace*{-0.8cm}
%     \begin{itemize}
%       \item Two types of passengers
%       \begin{itemize}
%         \item \HL{Punctual}, show up at pick-up location mostly ahead of pick-up time.
%         \item \HL{Tardy}, arrive closer to or even later than at the scheduled pick-up time.
%       \end{itemize}
%       \item Sources of randomness
%       \begin{itemize}
%         \item  Departure from home (or last destination) towards pick-up location (Normal Distribution).
%         \item Walking time (Truncated Normal Distribution).
%       \end{itemize}
%       % \item Population types
%       %   \begin{itemize}
%       %     \item \si{80\,\percent} punctual passengers.
%       %     \item \si{50\,\percent} punctual passengers.
%       %     \item \si{20\,\percent} punctual passengers.
%       %   \end{itemize}
%     \end{itemize}
%     \begin{figure}
%       \includegraphics[width=0.9\textwidth]{./figures/passengerProfiles.png}
%
%     \end{figure}
%   \end{frame}
%
%
%   \begin{frame}
%     \frametitle{Experimentation}
% %Implementation of simulations in  \csharp.
% We used the following setup for our simulation runs:
%     \begin{itemize}
%       \item \num{100} commuters, \num{10} vehicles.
%       \vspace*{0.2cm}
%       \item \num{10} samples, consider all vehicle transport schedules where all transport chains have been accepted.
%       \vspace*{0.2cm}
%
%       \item \num{100} simulation runs each ($\sim$ \num{5}\,seconds).
%       \vspace*{0.2cm}
%
%       \item Varying waiting time for late passengers $\omega$.
%     \end{itemize}
%   \end{frame}
%
%
%   \begin{frame}
%     \frametitle{Results: $\omega=0$ (no waiting for late passengers)  \RNum{1}}
%     \vspace*{-1cm}
%     \begin{figure}
%       \includegraphics[width=0.9\textwidth]{./figures/both.pdf}
%       % \caption{}
%       % \label{}
%     \end{figure}
%     \vspace*{-0.2cm}
%         \begin{itemize}
%     \item  More punctual passengers $\rightarrow$ earlier arrival.
%     \item  Good portion of the passengers arrive within the arrival time window.
%     \item  Arrival at destination is rarely more than $5$\,min later than planned arrival time window (given experimental population).
%     \end{itemize}
%   \end{frame}
%
%
%   \begin{frame}
%     \frametitle{Results: $\omega=0$ (no waiting for late passengers) \RNum{2}}
%     \footnotesize
%     \centering
%     % !TeX root = main.tex
% !TeX spellcheck = en_US

\begin{tabular}{
   S[table-format=2.0, table-column-width=1.5cm]
    |
S[table-format=2.2, table-column-width=1.5cm] 
 rrrrr
  }
  \hline
 {\punctual}    &    {$\ell >0$}    &  {$q_{25}$} &    {$q_{50}$}  &  {$q_{75}$} & {$q_{95}$}    &  {$q_{99}$}
 \\
   {(\%)}   &      {(\%)}    &  {(mm.ss)} &    {(mm.ss)}  &   {(mm.ss)} &  {(mm.ss)}    &   {(mm.ss)}
   \\
  \hline
 80  & 10.80466 & -13.53& -7.58 & -2.45 & 1.27 & 4.07  \\
  50  &  12.11865   & -13.20 & -7.31  & -2.15 & 1.34 & 4.10  \\
  20   & 13.26574 & -12.52 & -7.10  & -1.51 & 1.40 & 4.14  \\
  \hline
\end{tabular}
 % \maxf{-0.001589677}




%     \vspace*{1cm}
%     % !TeX root = main.tex
% !TeX spellcheck = en_US

\begin{tabular}{
   S[table-format=2.0, table-column-width=1.5cm]|
  S[table-format=2.2, table-column-width=2cm]
  S[table-format=2.2, table-column-width=2cm]
  S[table-format=2.2, table-column-width=2cm]
  }
  \hline
  {\punctual}     &    {aborted \outgoing}    &    {aborted \return}   & {both completed}\\
  {(\%)} &       {(\%)}       &  {(\%)}    &  {(\%)}  \\

  \hline
  80        & 3.56    & 1.82 &  94.6   \\
  50       & 4.74     &  3.06 & 92.2  \\
  20        & 5.83     & 4.22 &  90.0   \\
  \hline
\end{tabular}
 % \maxf{-0.001589677}
% 80        & 1339991  & 3.56    & 1.82 &  94.6   \\
 % 50        & 1339981 & 4.74     &  3.06 & 92.2  \\
 % 20        & 1339977 & 5.83     & 4.22 &  90.0   \\

%   \end{frame}
%
%
%   \begin{frame}
%     \frametitle{Results: $\omega=0,\ldots,10$\,min   \RNum{1}}
%     \vspace*{-0.4cm}
%
%     \begin{figure}
%       \includegraphics[width=1\textwidth]{./figures/aborted_differentWaitTimes.pdf}
%       % \caption{}
%       % \label{}
%     \end{figure}
%       \begin{itemize}
%     \item  Percentage of aborted requests goes down when introducing a waiting policy.
%     \item  Extensive waiting time $\omega$ leads to raising numbers again.
%     \item  Effect of  \HL{punctuality} is consistent over all experiments.
%     \end{itemize}
%   \end{frame}
%
%
%   \begin{frame}
%  %   \frametitle{Results: $\omega=0,\ldots,10$\,min   \RNum{2}}
% %\vspace*{-1.5cm}
% \vspace*{0.2cm}
%     \begin{figure}
%       \includegraphics[width=0.9\textwidth]{./figures/quantiles_h5.pdf}
%       % \caption{}
%       % \labelq{}
%     \end{figure}
%         \begin{itemize}
%     \item  Quantile values move (increase) with increasing waiting time $\omega$.
%     \item  This effect is rather weak.  Values do not increase by more than $1$\,min ($99\,\%$ quantile).
%     \end{itemize}
%   \end{frame}
%
%
%   \section{Conclusion}
%   \begin{frame}
%     % \frametitle{Conclusion}
%     \vspace*{1cm}
%     \HL{Conclusion}
%     \begin{itemize}
%       \item Analysis of real-world performance of transport schedules for
%       \textit{dial-a-ride} problem.
%       \item Agent-based modeling and simulation to execute the schedules under the influence
%       of disturbances.
%
%       \item Punctuality of passengers affects service provision.
%       \item Introducing a waiting policy can improve service.
%       \item Service providers can use the presented approach to fine-tune their policies.
%     \end{itemize}
%     \HL{Future Work}
%
%     \begin{itemize}
%       \item Explore relationship between late arrival of \HL{vehicles} and aborted requests
%       in more detail.
%       \item Apply approach to urban settings.
%     \end{itemize}
%   \end{frame}



  %%%%%%%%%%%%%%%%%%%%%%%%%%%%%%%%%%%%%%%%%%%%%%%%%%%%%%%%%%%%%%%%



  %%%REFERENCES
  %if no references used, comment this section

  % \appendix
  % \begin{frame}[allowframebreaks]
  %   \frametitle{References}
  %   \vspace*{-1cm}
  %   \tiny
  %   \bibliographystyle{abbrvnat}
  %   \bibliography{references}
  % \end{frame}


  \begin{frame}[plain]{\titlepage}\end{frame}
    \end{document}
