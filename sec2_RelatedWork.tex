%
% !TeX root =./main.tex
% !TeX spellcheck = en_US

Increasing economic development and changing economic conditions of countries rises the demand of OHC transportation. Taking into account the complexity of road transportation, proper freight transportation route selection is a critical, versatile task \cite{bazaras2013optimal, xu2001methodology, sivilevicius2007dynamics}. 
The presented research is built upon existing works on OHC literature. 

Lamirauxet al. present’s automated constant transportation speed security, Cheng and Lester analyse trajectory optimization, Durham, Faghri 2002; Datla et al. present’s GIS application to route planning and etc. However, the investigation of individual cases narrows the boundaries of each of the researches; thus, their results become hardly adaptable to other transportation conditions, that does not allow comparing different means of transport and providing analytical flexibility. The systems suggested by the authors could not be applied to planning the routes for other means of transport without significant changes.  Batarlienė, Kersten et al., Parentela, Rikov analyse transportation risks and the principles of the traffic intensity evaluation and planning, the mathematical dependencies, allowing rating and planning the levels of traffic intensity in certain road sections. These mathematical principles could be successfully used when planning the OHC transportation routes. Park, Woxenius, Shekharan, Ostrom, Rodrigue, Comtois are stressing on importance of planning a new road, or a reconstruction of an existing one, technical and economic calculations, basing the necessity and economic expediency of the road, should be taken into consideration. The calculation principles of the cargo traffic and cargo turnover suggested by the researchers could be applied when dealing with the same calculations in case of OHC  transportation. However, the cases of the multiple use of the road transport infrastructure for the OHC transportation should be taken into account. Janic, Kovalenko and study of Simplextrans are making emphasizes on the fact that global tendencies related to the increase of the rail freight efficiency by increasing the axial load and charging of the vehicle could easily solve the problem of the OHC transportation by rail. It could be assumed that the additional revenue obtained by maximising the efficiency of the rail system due to the cargo and OHC transportation by heavyweight trains could compensate the increased costs of the road infrastructure and vehicle repairs.
It should be noted that the rail track reconstruction or the construction of a new section for the OHC transportation could be expedient only in case of planning a permanent routes suitable for the OHC transportation. 
 It should be noted that the rail track reconstruction or the construction of a new section for the OHC transportation could be expedient only in case of planning a permanent routes suitable for the OHC transportation. The review of the literature sources has shown that the alternative for the combination of the rail and road means of transport for the transportation of the OHC has not been practically used. It is obvious that such a drawback is caused by the insufficient application of the rail infrastructure to the process of cargo handling, and the lack of the OHC reloading equipment, or its insufficient mobility. The latter statement is also applicable to the inland water transportation.

According to the works of Janic, Woxenius, Heatco studies dealing with water transport, this mean of transportation is the least sensitive to the load and mass of the cargo. In water transportation, the problem of the overweight of the cargoes is easily solved as both maritime and inland ports are usually the star and the end points of the OHC transportation route; thus the carriers are tend to use water, especially maritime, transportation. On the basis of the literature reviewed, it could be assumed that inland water transportation is more preferable for the OHC transportation due to lower restrictions for weight, dimensions, power consumption and accidents. The largest shortcoming of the latter mean of transport is related to the seasonality in the areas, in which water routes freeze in winter, and water level gets lower in summer. In the global transport market, especially within the EU, the transport policy attitude, stating that all cargoes transported by road should be transported by water or rail if it is possible, has recently prevailed. Considering to the possible transport development prospects, it could be pointed out that the latter trend will probably continue for a long time. Hence, when dealing with the methodologies and systems of the OHC transportation planning, the fact mentioned above should be taken into account. On the territories of individual countries or their economic unions, the tracks of the oversize and heavyweight cargoes should be considered the part of the system of the economic infrastructure of the territory in question. Such systems should be planned on the basis of the systematic interdisciplinary principles. The evaluation system for the OHC transportation processes, considering all mentioned factors, could solve the latter problem. 