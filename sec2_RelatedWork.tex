%
% !TeX root =./main.tex
% !TeX spellcheck = en_US

Increasing economic development and changing economic conditions of countries rises the demand of OHC transportation. Taking into account the complexity of road transportation, proper freight transportation route selection is a critical, versatile task \cite{Bazaras.2013, xu2001methodology, sivilevicius2007dynamics}. In general, designing a reasonable transportation scheme can greatly improve the efficiency of OHC transport \cite{meng2015optimized}. Road transportation as the main means of transportation for OHC is very flexible. The goods can thus be transported directly to their destination avoiding secondary transports (water, rail), thus saving transport costs. In addition, it provides a higher level of safety and reliability of transportation. The optimization of route selection for the transportation of OHC on road networks has currently become a topic of practical importance\cite{geisberger2011efficient}.
\par The presented research is built upon existing works on OHC literature dealing with different approaches and transportation issues to determine the route of OHC. Some examples of the analysis and planning of OHC transportation processes are presented; e.g., to estimate potential transportation routes of OHC \cite{durham2002gis} investigate the issue through the application of Geographic Information System (GIS). Datla et al. \cite{datla2004gis} states that using GIS software aims to find the shortest path in the existing road network taking into account the vehicle’s height and weight for each road segment. However, from a practical point of view, it is very important that OHC transports do not deviate from the planned transport route. This planned route is usually chosen by evaluating the vehicle parameters and dimensions together with the load and weight distribution and other characteristics to ensure road safety or prevent damage to road infrastructure, e.g. bridges \cite{ecmt2006improving, vaitkus2016effect, kombe2017modelling, pauer2017development}.
Others such as Batarlienė \cite{batarliene2007mobile} and Parentela \cite{parentela2002risk} analyse transportation risks and the principles of the traffic intensity evaluation and planning, the mathematical dependencies, allowing rating and planning the levels of traffic intensity in certain road sections. These mathematical principles could be successfully used when planning the OHC transportation routes. Woxenius \cite{woxenius2002organisation, woxenius2002conceptual}, Rodrigue \cite{rodrigue2020geography}are stressing on importance of planning a new road, or a reconstruction of an existing one, technical and economic calculations, basing the necessity and economic expediency of the road, should be taken into consideration. The calculation principles of the cargo traffic and cargo turnover suggested by the researchers could be applied when dealing with the same calculations in case of OHC  transportation. However, the cases of the multiple use of the road transport infrastructure for the OHC transportation should be taken into account. 

The literature review has shown that considering bridge capacities as a limiting factor for OHC route selection has not been scientifically analyzed so far. It is obvious that such a drawback is caused by the insufficient information availability of the road network infrastructure to the process of OHC handling. Hence, when dealing with the methodologies and systems of the OHC transportation planning, the fact mentioned above should be taken into account. On the territories of individual countries or their economic unions, the tracks of the oversize and heavyweight cargoes should be considered the part of the system of the economic infrastructure of the territory in question. Such systems should be planned on the basis of the systematic interdisciplinary principles. The evaluation system for the OHC transportation processes, considering all mentioned factors, could solve the latter problem. 