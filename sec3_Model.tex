%
% !TeX root =./main.tex
% !TeX spellcheck = en_US
\subsection{Matrix Representation of the Road Network}

For now, we consider the \textit{higher level road network} in Carinthia, Austria,
i.e., \textit{A}, \textit{S}, \textit{L}, \textit{B}, in the  form of a matrix.
The vertices of the matrix are the crossing points of the roads, while
the road section connecting those points form the edges of the matrix.
For each edge, we have the distance between the vertices and the length of the bridges along the segment
and the capacity constraints (and the lowest encountered bridge capacity).
This way, we can answer various questions concerning transports between those vertices.
Without loss of generality, we assume an undirected graph.
In that sense, the travel distance along an edge is independent of
the travel direction, and  the same bridges are surpassed in both directions.
The problem statement can
easily be extended to a directed graph. However, the authors omitted this for reasons of simplicity.

For now, we explore the different objectives on a simple toy example that we illustrate in Figure \ref{fig_toy_example_1}.


\begin{figure}[!ht]
 \centering
  \includegraphics[width=0.9\textwidth]{map.jpg}
  \caption{Overview of the higher level road network in Carinthia.}
  \label{fig:higher level}
\end{figure}


\begin{figure}[!ht]
  \centering
  % !TeX root =../main.tex
% !TeX spellcheck = en_US

% https://tex.stackexchange.com/questions/64252/tikz-midway-label-on-a-bended-line
% https://texample.net/tikz/examples/p2p-topology/

\begin{tikzpicture}[auto, thick]

  % Define Colors
  \definecolor{pinegreen}{cmyk}{0.92,0,0.59,0.25}
  \definecolor{royalblue}{cmyk}{1,0.50,0,0}
  \definecolor{lavander}{cmyk}{0,0.48,0,0}
  \definecolor{violet}{cmyk}{0.79,0.88,0,0}
  \definecolor{red}{cmyk}{0,0.95,0.90,0}
  \definecolor{yellow}{cmyk}{0,0.25,1,0}

  % Node styles
  \tikzstyle{cblue}=[circle, draw, thin,fill=cyan!20, scale=0.8]
  \tikzstyle{qgre}=[rectangle, draw, thin,fill=green!20, scale=0.8]

  \tikzstyle{A_path}=[ultra thick, red, opacity=0.8, font=\small]
  \tikzstyle{S_path}=[ultra thick, yellow, opacity=0.8, font=\small]
  \tikzstyle{B_path}=[ultra thick, royalblue, opacity=0.8, font=\small]
  \tikzstyle{L_path}=[ultra thick, pinegreen, opacity=0.8, font=\small]


  % Nodes
  % \node[cblue] (n_1_1) at (0,0) {1};
  % \node[cblue] (n_2_1) at (4,0) {2};
  \node[cblue] (n_6) at (8,0) {6};

  % \node[cblue] (n_1_2) at (0,4) {4};
  \node[cblue] (n_4) at (4,4) {4};
  \node[cblue] (n_5) at (8,4) {5};

  \node[cblue] (n_1) at (0,8) {1};
  \node[cblue] (n_2) at (4,8) {2};
  \node[cblue] (n_3) at (8,8) {3};




  %  A links

  \draw[A_path,postaction={decorate,decoration={text along path,text align=center,text={$d=3, ~w=10$},raise=-10pt}}](n_4)--  (n_5)  node [midway, above, sloped] {$[10t]$};
  \draw[A_path,postaction={decorate,decoration={text along path,text align=center,text={$d=3.7, ~w=10$},raise=-10pt}}] (n_6) to[bend left]   node [midway, above, sloped]  {$[10t]$} (n_5)  ;
  \draw[A_path, postaction={decorate,decoration={text along path,text align=center,text={$d=4, ~w=10$},raise=-10pt}}] (n_1) -- (n_4)  node [midway, above, sloped]  {$[10t]~[10t]$};


  % S links
  \draw[S_path, postaction={decorate,decoration={text along path,text align=center,text={$d=8, ~w=10$},raise=-10pt}}] (n_4) to node [midway, above, sloped] {$[10t]$} (n_6);

  %  L lines
  \draw[L_path, postaction={decorate,decoration={text along path,text align=center,text={$d=4, ~w=10$},raise=-10pt}}] (n_1) --  (n_2)  node [midway, above, sloped] (TextNode) {$[10t]$};

  \draw[L_path, postaction={decorate,decoration={text along path,text align=center,text={$d=5$},raise=-10pt}}] (n_2)--  (n_3)  node [midway, above, sloped] (TextNode) {};
  % \draw[L_path] (n_3) to[out=-20,in=-20]  (n_5)  node [midway, above, sloped] (TextNode) {$l=3, b=2$};
  \draw[L_path,postaction={decorate,decoration={text along path,text align=center,text={$d=3, ~w=10$},raise=-10pt}}] (n_3) to[bend left] node [midway, above, sloped]   {$[10t]$}    (n_5);

  \draw[L_path, postaction={decorate,decoration={text along path,text align=center,text={$d=4.5$},raise=-10pt}}] (n_5) to[bend left]  node [midway, above, sloped] {} (n_6);

  % B links
  % \draw[B_path] (n_6) to node [midway, above, sloped] {$7; 2$} (n_2_1);


  % Legends
  % \node[A_path, anchor=west] at (-3,8){\textsc{A:} $w=2^0$};
  % \node[S_path,anchor=west] at (-3, 7.5){\textsc{S:} $w=2^1$};
  % \node[L_path,anchor=west] at (-3, 7){\textsc{L:} $w=2^2$};
  % \node[B_path,anchor=west] at (-3, 6.5){\textsc{B:} $w=2^3$};

  \node[ anchor=west] at (0,2){Levels: {\color{red}\textsc{A}}, {\color{yellow}\textsc{S}},  {\color{royalblue}\textsc{B}},  {\color{pinegreen}\textsc{L}}  } ;


\end{tikzpicture}

  \caption{Toy Example 1.}
  \label{fig_toy_example_1}
\end{figure}
