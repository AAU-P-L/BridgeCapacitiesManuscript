%
% !TeX root =./main.tex
% !TeX spellcheck = en_US
\subsection{Matrix Representation of the Road Network}

For now, we consider the \textit{higher level road network} in Carinthia, Austria,
i.e., \textit{A}, \textit{S}, \textit{L}, \textit{B}, in the  form of a matrix.
The vertices of the matrix are the crossing points of the roads, while
the road section connecting those points form the edges of the matrix.
For each edge, we have the distance between the vertices and the length of the bridges along the segment
and the capacity constraints (and the lowest encountered bridge capacity).
This way, we can answer various questions concerning transports between those vertices.
Without loss of generality, we assume an undirected graph.
In that sense, the travel distance along an edge is independent of
the travel direction, and  the same bridges are surpassed in both directions.
The problem statement can
easily be extended to a directed graph. However, the authors omitted this for reasons of simplicity.

Classically,  the considered network $\mathcal{N}=(\mathcal{V},\mathcal{E})$ consists of
\begin{itemize}
  \item \emph{Nodes} (vertices) $\mathcal{V}=\{1,\ldots, n\}$, which represent the intersections of road
  well as the start (end) points of the roads, and

  \item \emph{Links} (edges) $\mathcal{E} \subseteq \mathcal{L} \times \mathcal{L}$,
  correspond to the roads (road segements) connecting the vertices \cite{liedtke2012generation}.
\end{itemize}

Road links $e \in \mathcal{E}$ have a length $\ell(e)\in \mathbb{R}^{+}$ representing their road length.
Further, they have a weights $c(e) \in C$ giving their \emph{classification} following \textit{Open Street Map} (OSM) \cite{OpenStreetMap}. Starting from the highest level, we classify the roads
$C=\{0,1,2\,\ldots\}$ where these are regarded dimensionless units.

For each road link  $e \in \mathcal{E}$ we know the \emph{maximum gradient} (descent and ascent). We denote them by $g^{+}(e) \in [0,1)$ and  $g^{-}(e) \in (-1,0]$ for the steepest ascent and descent, accordingly.

Bridges (and other types of building structures that must be overpassed)
have \emph{weight limits} (measured in metric tons) which are determined by civil engineers.
Each road link $e \in \mathcal{E}$ contains a nonnegative number of bridges, where each has a weight limit.
For each link $e$, we set a \emph{weight limit} $w(e) \in \mathbb{R}^{+}$ which is the
minimum over all bridges on this link and the \emph{number of bridges} $b(e) \in \mathbb{N}^{0}$.
A weight limit $w(e)= \infty$ describes the circumstance
that no limit is given for $e$.
Similarly, we assume that nodes $v \in \mathcal{V}$ can have weight limits.
\begin{figure}[!ht]
  \centering
  \includegraphics[width=0.9\textwidth]{map.jpg}
  \caption{Overview of the higher level road network in Carinthia.}
  \label{fig:higher level}
\end{figure}


A simple toy example, see Figure \ref{fig_toy_example_1}, shall illustrate these definitions.
\begin{figure}[!ht]
  \centering
  % !TeX root =../main.tex
% !TeX spellcheck = en_US

% https://tex.stackexchange.com/questions/64252/tikz-midway-label-on-a-bended-line
% https://texample.net/tikz/examples/p2p-topology/

\begin{tikzpicture}[auto, thick]

  % Define Colors
  \definecolor{pinegreen}{cmyk}{0.92,0,0.59,0.25}
  \definecolor{royalblue}{cmyk}{1,0.50,0,0}
  \definecolor{lavander}{cmyk}{0,0.48,0,0}
  \definecolor{violet}{cmyk}{0.79,0.88,0,0}
  \definecolor{red}{cmyk}{0,0.95,0.90,0}
  \definecolor{yellow}{cmyk}{0,0.25,1,0}

  % Node styles
  \tikzstyle{cblue}=[circle, draw, thin,fill=cyan!20, scale=0.8]
  \tikzstyle{qgre}=[rectangle, draw, thin,fill=green!20, scale=0.8]

  \tikzstyle{A_path}=[ultra thick, red, opacity=0.8, font=\small]
  \tikzstyle{S_path}=[ultra thick, yellow, opacity=0.8, font=\small]
  \tikzstyle{B_path}=[ultra thick, royalblue, opacity=0.8, font=\small]
  \tikzstyle{L_path}=[ultra thick, pinegreen, opacity=0.8, font=\small]


  % Nodes
  % \node[cblue] (n_1_1) at (0,0) {1};
  % \node[cblue] (n_2_1) at (4,0) {2};
  \node[cblue] (n_6) at (8,0) {6};

  % \node[cblue] (n_1_2) at (0,4) {4};
  \node[cblue] (n_4) at (4,4) {4};
  \node[cblue] (n_5) at (8,4) {5};

  \node[cblue] (n_1) at (0,8) {1};
  \node[cblue] (n_2) at (4,8) {2};
  \node[cblue] (n_3) at (8,8) {3};




  %  A links

  \draw[A_path,postaction={decorate,decoration={text along path,text align=center,text={$d=3, ~w=10$},raise=-10pt}}](n_4)--  (n_5)  node [midway, above, sloped] {$[10t]$};
  \draw[A_path,postaction={decorate,decoration={text along path,text align=center,text={$d=3.7, ~w=10$},raise=-10pt}}] (n_6) to[bend left]   node [midway, above, sloped]  {$[10t]$} (n_5)  ;
  \draw[A_path, postaction={decorate,decoration={text along path,text align=center,text={$d=4, ~w=10$},raise=-10pt}}] (n_1) -- (n_4)  node [midway, above, sloped]  {$[10t]~[10t]$};


  % S links
  \draw[S_path, postaction={decorate,decoration={text along path,text align=center,text={$d=8, ~w=10$},raise=-10pt}}] (n_4) to node [midway, above, sloped] {$[10t]$} (n_6);

  %  L lines
  \draw[L_path, postaction={decorate,decoration={text along path,text align=center,text={$d=4, ~w=10$},raise=-10pt}}] (n_1) --  (n_2)  node [midway, above, sloped] (TextNode) {$[10t]$};

  \draw[L_path, postaction={decorate,decoration={text along path,text align=center,text={$d=5$},raise=-10pt}}] (n_2)--  (n_3)  node [midway, above, sloped] (TextNode) {};
  % \draw[L_path] (n_3) to[out=-20,in=-20]  (n_5)  node [midway, above, sloped] (TextNode) {$l=3, b=2$};
  \draw[L_path,postaction={decorate,decoration={text along path,text align=center,text={$d=3, ~w=10$},raise=-10pt}}] (n_3) to[bend left] node [midway, above, sloped]   {$[10t]$}    (n_5);

  \draw[L_path, postaction={decorate,decoration={text along path,text align=center,text={$d=4.5$},raise=-10pt}}] (n_5) to[bend left]  node [midway, above, sloped] {} (n_6);

  % B links
  % \draw[B_path] (n_6) to node [midway, above, sloped] {$7; 2$} (n_2_1);


  % Legends
  % \node[A_path, anchor=west] at (-3,8){\textsc{A:} $w=2^0$};
  % \node[S_path,anchor=west] at (-3, 7.5){\textsc{S:} $w=2^1$};
  % \node[L_path,anchor=west] at (-3, 7){\textsc{L:} $w=2^2$};
  % \node[B_path,anchor=west] at (-3, 6.5){\textsc{B:} $w=2^3$};

  \node[ anchor=west] at (0,2){Levels: {\color{red}\textsc{A}}, {\color{yellow}\textsc{S}},  {\color{royalblue}\textsc{B}},  {\color{pinegreen}\textsc{L}}  } ;


\end{tikzpicture}

  \caption{Toy Example 1.}
  \label{fig_toy_example_1}
\end{figure}


\subsection{Related Work on the Shortest Path Problem}

\citet{TACCARI2016122, zhu2014vehicle, Osegueda.1999}

\subsection{Mathematical Model}

Next, we give an exact problem formulation.

\subsubsection{Feasible Subnetwork}
A \emph{transport request} $r=(o,d,w,g^{-},g^{+}) \in \mathcal{V} \times \mathcal{V} \times \mathbb{R}^{+}$
has an \emph{origin} $s$, a \emph{destination} $t$, a \emph{total weight} $w$,
a \emph{maximal descent gradient} $g^{-}$, and a \emph{maximal ascent gradient} $g^{+}$.
The maximal gradients define the gradient that is doable for the transport with
creating a safety hazard.

Clearly, a transport $r$ can only pass through nodes $v \in \mathcal{V}$ with
$w(r) \leq w(v)$ and traverse road links $e \in \mathcal{E}$ for which $w(r) \leq w(e)$ holds.
Further, $r$ can only travel on road links $e$ for which $g^{-}(r)\leq g^-(e)$ and
 $g^{+}(r)\geq g^+(e)$ holds.

Accordingly, we define the sub-network $\mathcal{N}_{\omega}$ that contains only the vertices and edges from $\mathcal{N}$ that have a weight limit larger than $\omega \in \mathbb{R}^{+}$ and
does not violate the maximal gradients.
Hence, for the remainder of this manuscript, we set $\mathcal{N}=\mathcal{N}_{\omega=w(r)}$.
In that sense, a transport request $r$ can not be conducted if $\mathcal{N}_{\omega=w(r)}$ is not
connected between $o$ and $d$.

\paragraph{Example}
We consider a request $r=(o,d,w=20t, g^{-}=-0.10, g^{+}=0.08)$ where $o$ and $d$ are somewhere within the network. A road link $e_1=(o_1,d_1, w=30t, g^{-}=-0.08, g^{+}=0.05)$  can be traversed as, both, the bridge weight capacities and the gradients are feasible. On the other hand,  a link
$e_2=(o_2,d_2, w=10t, g^{-}=-0.12, g^{+}=0.07)$ cannot be used as there is at least one bridge
on the link that carries only $10t$ (while the transport has $20t$). Also, there
is a $-0.12$ descent ($-12\,\%$) that cannot be dealt with in a safely manner as the
transport is cannot handle more than $8\%$ descent.


\subsubsection{Integer Programming Formulation}
In order to give an \emph{Integer Programming} (IP) formulation for we enumerate the
vertices $1,\ldots,|\mathcal{V}|$. We consider a directed graph $G=(\mathcal{V},\mathcal{E})$ that is derived from the network $\mathcal{N}$.
In that sense, the auxiliary variables $x_{ij}$, $(i,j) \in \mathcal{V}$
is set to $1$ if and only if node $j$ is visited directly after node $i$, and to $0$ otherwise.
A standard IP formulation to determine the shortest path between $s$ and $t$ can be given as follows
\begin{align}
  \min \quad &\sum_{(i,j)\in \mathcal{E}}  c_{ij} x_{ij} \label{obj} \\
  \text{s.t.}\quad &
  \sum_{(i,j)\in \delta^{+} (i)} x_{ji} - \sum_{(i,j)\in \delta^{-}(i)} x_{ij} =
  \begin{cases}
    1 \quad& \text{if}~ i=s, \\
    -1 \quad& \text{if}~ i=t \\
    0 \quad&\text{else}
  \end{cases}
  \qquad \forall i \in \mathcal{E}
  \\
  &  \sum_{(i,j)\in \delta^{+} (i)} x_{ij}   x_ {ij} \leq 1     \qquad \forall i \in \mathcal{E}\\
  &  x_{ij} \in \{0,1\}   \qquad \forall (i,j) \in \mathcal{V}
\end{align}
The set of outgoing (ingoing) edges of vertex $i$ is denoted by  $\delta^{+} (i)$  $(\delta^{-} (i))$.


We consider three different objectives, i.e., different assignment of $c_{ij}$.
\begin{enumerate}
  \item \emph{Classic shortest path}, where $c_{ij}=\ell_{ij}$. \label{obj_short}
  \item \emph{Minimal number of surpassed bridges}, where  $c_{ij}=b_{ij}$.  \label{obj_minBridge}
  \item \emph{Prefer high level roads}, where we assign an exponential weighting to the road classes (in ascending order of preference) $C$, $c_0=2^0,c_1=2^1, \ldots$.  \label{obj_highLevelRoad}
  \item \emph{Avoid steep roads}, exponential weights are assigned to the roads.
  Therefore we summarize the gradients of a road segment $e$ into $g(e)=\max\{{|g^{-}(e)|+g^{+}(e)}\}$
  and assign them exponential weights $c_{ij}=\euler^{g(e)}$.
  \label{obj_steep}
  \item \emph{Minimize elevation}, minimize the total elevation that the transport
  must overcome. Sum of the total (descent) elevation of all links. $c_{ij}=\text{elevation}$.
\end{enumerate}

\subsubsection{Solution Procedere}

As the considered objectives use costs which induce no \emph{negative cycles}  on $G$, the problem can be efficiently solved
by polynomial-time algorithms like
\emph{Bellman-Ford's} \cite{Bell1958,Ford1956} or \emph{Dijistra's} \cite{dijkstra1959note} algorithm.
% In case of \ref{obj:obj_minBridge}, links having no bridges ($b(e)=0$) can induce cycles which are of length zero.


% bjectives are:
% \begin{itemize}
%   \item Shortest Path (classic).
%
%   \item Minimal wear of infrastructure. Reducing the wear induced by over-weight transports
%   minimizes maintenance costs, extends lifetime of the building structures, and
%   improves safety (Genoa bridge collapse in 2018).
%   \citet{Kakan2014}
%
%   % \item Minimize the number of different road operators and municipalities the path
%   % traverses. In that sense, the process of getting official approval of the
%   % path should be simplified.
%
%   \item Number of Bridges, i.e., less work for the civil engineer.
%   \item Preferably high level roads (if possible).
%   Achieve this be exponential weightening of the road segments.
%   \textit{A}: $w=2^0$,    \textit{S}: $w=2^1$,      \textit{B}: $w=2^2$,     \textit{L}: $w=2^3$.
%   The considered Austrian road network is clearly structured into these road levels.
%   Clearly, for any other network, a similar weightening be introduced
%   (if not given) under consideration of the road widths etc.
% \end{itemize}
%
% \begin{itemize}


% https://algs4.cs.princeton.edu/44sp/
% https://jgrapht.org/javadoc-SNAPSHOT/org.jgrapht.core/org/jgrapht/alg/shortestpath/DijkstraShortestPath.html
