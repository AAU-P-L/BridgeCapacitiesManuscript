%
% !TeX root =./main.tex
% !TeX spellcheck = en_US

Increasing economic development, general technological progress, and the growing prevalence of large-scale facilities, such as wind mills or power plants, increase the demand for OHC transports.
Frequently, these transport processes are executed within the road network. This is because roads exhibit a high geographical coverage and thus, provide a certain amount of flexibility and cost saving potential, as goods can often be delivered directly without any time-consuming handling or transshipment efforts being required. However, taking into account the complexity of road transportation as well as the potential damage caused to pavements and bridges, OHC route selection is a highly challenging task \cite{Bazaras.2013, xu2001methodology, sivilevicius2007dynamics, fiorillo2016minimizing}. Yet, selection of an optimal route based on reasonable route planning methods greatly improves the efficiency of OHC transports \cite{meng2015optimized}.
Therefore, optimal route selection in the field of road-related OHC has recently become a topic of considerable relevance \cite{geisberger2011efficient, yan2018optimal}.
\par
This paper is building upon past research on OHC routing. In the literature, a wide range of different approaches to this issue are presented. Fu and Hag-Elsafi \cite{fu2000vehicular}, for example, address the overload monitoring and permitting procedure. In doing so, reliability models for assessing bridge safety are described. Moreover, permit-load factors for monitoring purposes are proposed. Ghosn, in turn, focuses on truck weight regulations and the corresponding bridge safety levels using reliability indices \cite{ghosn2000development}. Vigh and Kollár, again, elaborate on methods that take into account different bridge structures as well as vehicle and loading characteristics in the permitting process \cite{vigh2006approximate, vigh2007routing}
\par
One particular method for optimal route selection is the integration and processing of all available data concerning the road network in a Geographic Information System (GIS) \cite{durham2002gis}.
According to Datla et al. \cite{datla2004gis}, a GIS facilitates identification of shortest paths, while taking into account characteristics and attributes of the road network as well as the involved transport vehicles, provided that data are available and up-to-date. Relevant parameters, such as vehicles' load, height, width, weight, and weight distribution, are considered to take into account restrictions, ensuring safety, and preventing damage to infrastructure elements such as bridges \cite{ecmt2006improving, vaitkus2016effect, kombe2017modelling, pauer2017development}. According to Adams et al. \cite{adams2002enterprise}, a GIS-based system also facilitates the automation of the permission process by considering spatial and temporal constraints during the path-finding process, given that the respective systems for routing and authorization purposes are well integrated. However, the authors also emphasize the difficulty of aligning enterprise databases of bridges and highways with the requirements of GIS-based permitting systems, which is frequently resulting in severe data management problems.
\par
According to Bazaras et al. \cite{Bazaras.2013}, safety, security, and reliability are indeed three very important aspects that have to be considered to increase the overall quality of transport processes. Therefore, apart from regular driving training and state-of-the-art equipment, detailed risk evaluation is regarded to be a key issue in terms of OHC routing. The authors also underline that a fuzzy multi-criteria decision making tool is beneficial in the course of route selection, as it allows taking into account the variety of influencing factors. Apart from road quality, turning radius, corridor widths and heights, bridge carrying capacities, reloading and storing opportunities, regular traffic intensity on route segments, and seasonal specialties, a multi-criteria system might also include an assessment of required road texture improvements and other structural adaptations or changes.
\par
In summary, the respective literature does not take adequate account of bridge carrying capacities as an important influencing factor for OHC route selection. We assume the underlying reason being a general lack of data and insufficient data quality on exact bridge locations and carrying capacities.



% \begin{itemize}
%
%   \item Commercial Solutions
%
%   \item HERE
%   \url{https://www.here.com/}
%   \item PRISMA
%   \url{https://www.prisma-solutions.com/}
%
%   \item Bentley Superload Routing
%   \url{https://www.bentley.com/en/products/product-line/asset-performance/superload-routing}
%
% \end{itemize}
