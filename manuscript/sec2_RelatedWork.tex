%
% !TeX root =./main.tex
% !TeX spellcheck = en_US

Increasing economic development, general technological progress, and the growing prevalence of large-scale facilities, such as wind mills or power plants, increase the demand for OHC transports.
Frequently, these transport processes are executed within the road network. This is because roads exhibit a high geographical coverage and thus, provide a certain amount of flexibility and cost saving potential, as goods can often be delivered directly without any time-consuming handling or transshipment efforts being required. However, taking into account the complexity of road transportation, OHC route selection is a challenging task \cite{Bazaras.2013, xu2001methodology, sivilevicius2007dynamics}.
Yet, selection of an optimal route based on reasonable route planning methods greatly improves the efficiency of OHC transports \cite{meng2015optimized}.
Therefore, optimal route selection in the field of road-related OHC has recently become a topic of high relevance \cite{geisberger2011efficient}.
\par
This paper is building upon past research on OHC routing. In the literature, a wide range of different approaches to this issue are presented.
One particular method for optimal route selection is the integration and processing of all available data concerning the road network in a Geographic Information System (GIS) \cite{durham2002gis}.
According to Datla et al. \cite{datla2004gis}, a GIS facilitates identification of shortest paths, while taking into account characteristics and attributes of the road network as well as the involved transport vehicles, provided that data are available and up-to-date. Relevant parameters, such as vehicles' load, height, width, weight, and weight distribution, are considered to take into account restrictions, ensuring safety, and preventing damage to infrastructure elements such as bridges \cite{ecmt2006improving, vaitkus2016effect, kombe2017modelling, pauer2017development}.
\par
In summary, the respective literature does not take adequate account of bridge capacities as an important influencing factor for OHC route selection. We assume the underlying reason being a general lack of data and insufficient data quality on exact bridge locations and carrying capacities.



% \begin{itemize}
%
%   \item Commercial Solutions
%
%   \item HERE
%   \url{https://www.here.com/}
%   \item PRISMA
%   \url{https://www.prisma-solutions.com/}
%
%   \item Bentley Superload Routing
%   \url{https://www.bentley.com/en/products/product-line/asset-performance/superload-routing}
%
% \end{itemize}
