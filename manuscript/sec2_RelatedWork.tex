%
% !TeX root =./main.tex
% !TeX spellcheck = en_US

Increasing economic development and changing economic conditions of countries rises the demand of OHC transportation.
Taking into account the complexity of road transportation, proper freight transportation route selection is a critical, versatile task \cite{Bazaras.2013, xu2001methodology, sivilevicius2007dynamics}.
 In general, designing a reasonable transportation scheme can greatly improve the efficiency of OHC transport \cite{meng2015optimized}.
 Road transportation as the main means of transportation for OHC is very flexible.
  The goods can thus be transported directly to their destination avoiding secondary transports (water, rail), thus saving transport costs.
   In addition, it provides a higher level of safety and reliability of transportation.
   The optimization of route selection for the transportation of OHC on road networks has currently become a topic of practical importance\cite{geisberger2011efficient}.
\par
 The presented research is built upon existing works on OHC literature dealing with different approaches and transportation issues to determine the route of OHC.
 Some examples of the analysis and planning of OHC transportation processes are presented; e.g., to estimate potential transportation routes of OHC \cite{durham2002gis} investigate the issue through the application of Geographic Information System (GIS).
 Datla et al. \cite{datla2004gis} states that using GIS software aims to find the shortest path in the existing road network taking into account the vehicle’s height and weight for each road segment. However, from a practical point of view, it is very important that OHC transports do not deviate from the planned transport route.
  This planned route is usually chosen by evaluating the vehicle parameters and dimensions together with the load and weight distribution and other characteristics to ensure road safety or prevent damage to road infrastructure, e.g. bridges \cite{ecmt2006improving, vaitkus2016effect, kombe2017modelling, pauer2017development}.
Others such as Batarlienė \cite{batarliene2007mobile} and Parentela \cite{parentela2002risk} investigate different aspects of transportation risks and traffic intensity evaluation via mathematical dependencies that allow the planning of traffic intensity in certain road sections.
In general, mathematical prinicples are beneficial when planning OHC transportation routes.
Woxenius \cite{woxenius2002organisation, woxenius2002conceptual}, Rodrigue \cite{rodrigue2020geography} state that, during the planning of a new road, or the reconstruction of an existing one, technical and economic calculations must be carried out.
These calculations must be carried out considering the economic convenience of the road.
The calculation fundamentals of freight transport and cargo handling proposed by these researchers could be applied when dealing with the same calculations in the case of OHC transportation. However, the situations of multiple uses of road transportation infrastructure for OHC carriage should be taken into account.

The literature review has shown that considering bridge capacities as a limiting factor for OHC route selection has not been scientifically analyzed so far.
It is obvious that such a drawback is caused by the insufficient information availability of the road network infrastructure to the process of OHC handling.
Therefore, addressing the methodologies and systems of OHC transportation planning, the above-mentioned circumstance should be considered.
The routes eligible for oversized and heavy cargoes should always be considered as part of the system of economic infrastructure of the respective economic territory.

GIS-based routing einfügen
GIS-based software etc.


% \begin{itemize}
%
%   \item Commercial Solutions
%
%   \item HERE
%   \url{https://www.here.com/}
%   \item PRISMA
%   \url{https://www.prisma-solutions.com/}
%
%   \item Bentley Superload Routing
%   \url{https://www.bentley.com/en/products/product-line/asset-performance/superload-routing}
%
% \end{itemize}
