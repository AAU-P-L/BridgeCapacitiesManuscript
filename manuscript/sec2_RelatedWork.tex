%
% !TeX root =./main.tex
% !TeX spellcheck = en_US

Increasing economic development, general technological progress, and the growing prevalence of large-scale facilities, such as wind mills or power plants, increase the demand for OHC transports.
Frequently, these transport processes are executed within the road network. This is because roads exhibit a high geographical coverage and thus, provide a certain amount of flexibility and cost saving potential, as goods can often be delivered directly without any time-consuming handling or transshipment efforts being required. However, taking into account the specific requirements of OHC transports and the associated risks as well as the potential damage caused to pavements and bridges, OHC route selection is a highly challenging task \cite{Bazaras.2013, xu2001methodology, sivilevicius2007dynamics, fiorillo2016minimizing, wu2019assessment, lou2016effect}. In addition, the course of action from drafting the route until a permit is granted is costly and time-consuming. This is because civil engineers need to evaluate whether or not a proposed route is feasible. Thus, route planning based on sophisticated methods not only improves the efficiency of the underlying transport process but also increases the likelihood of obtaining immediate approval \cite{meng2015optimized}. For this reason, routing solutions that are aligned to the distinctive requirements of road-related OHC transportation have recently become a topic of considerable relevance \cite{geisberger2011efficient, yan2018optimal, zhu2014vehicle}.
\par
This paper is building upon past research on OHC routing. Two weight-related parameters, i.e., road gradients and bridge carrying capacities, draw our particular attention because we regard them to be highly safety-relevant. While bridge carrying capacities are occasionally addressed in the OHC routing literature, topography is only considered marginally. Topography as a relevant influencing factor can rather be found in research studies on route planning for electric vehicles \cite{perger2020energy}, path optimization in traditional logistics systems \cite{dang2019cost}, and people's general route planning and wayfinding behavior \cite{brunye2015strategies}. In their study on route planning for electric vehicles, Perger and Auer \cite{perger2020energy} evaluate the impact of topography on energy consumption, battery lifetime, and journey time as well as they address in detail the issue of modeling the street network with complex properties such as slope. In their study on path optimization in traditional logistics systems, Dang et al. \cite{dang2019cost} underline the importance of taking into account the influence of different loading levels and road gradients as well as they point out safety issues and excessive vehicle wear when steep slopes coincide with heavy load. According to Brunyé et al. \cite{brunye2015strategies}, topography influences route selection especially in unfamiliar environments and mountainous regions. 
\par
In terms of bridge carrying capacities, a wide range of different approaches are presented in the literature. Fu and Hag-Elsafi \cite{fu2000vehicular}, for example, address the overload monitoring and permitting procedure. In doing so, reliability models for assessing bridge safety are described. Moreover, permit-load factors for monitoring purposes are proposed. Ghosn \cite{ghosn2000development}, in turn, focuses on truck weight regulations and the corresponding bridge safety levels using reliability indices. By using weigh-in-motion technology, Gungor et al. \cite{gungor2018detect} develop a fully data-driven framework that considers individual truck loading levels as well as respective bridge carrying capacities to compute fees for overweight vehicles passing bridges. The framework can be integrated in existing online permit issuing tools. Lou et al. \cite{lou2016effect} aim at quantifying the the negative effect of overweight trucks on bridges via deterioration models. Vigh and Kollár \cite{vigh2006approximate, vigh2007routing}, again, elaborate on methods that take into account different bridge structures as well as vehicle and loading characteristics in the permitting process. Apart from total loading weight or vehicle and loading characteristics, regulations often target at the transverse truck position or the speed limit when crossing bridges \cite{yan2018optimal}. Zhu et al. \cite{zhu2014vehicle} establish a constrained maximum-capacity linear-integer programming model that is suitable for real-time optimization. The procedure is based on a shortest paths algorithm and considers limitations, such as prohibitive turns, limited bridge carrying capacities, narrow roads, and low underpasses, while minimizing fuel consumption. However, one deficiency of the study is that capacities are assigned randomly.
\par
In terms of general route planning, one particular method is the integration and processing of all available data in a geographic information system (GIS) \cite{durham2002gis, dayan2014methodology}.
According to Datla et al. \cite{datla2004gis}, a GIS facilitates identification of shortest paths, while taking into account characteristics and attributes of the road network as well as the involved transport vehicles, provided that data are available and up-to-date. Relevant parameters, such as vehicles' load, height, width, weight, and weight distribution, are considered to take into account restrictions, ensuring safety, and preventing damage to infrastructure elements such as bridges \cite{ecmt2006improving, vaitkus2016effect, kombe2017modelling, pauer2017development}. According to Adams et al. \cite{adams2002enterprise}, a GIS-based system also enables the automation of the permission process by considering spatial and temporal constraints during the path-finding process, given that the respective systems for routing and authorization purposes are well integrated. However, the authors also emphasize the difficulty of aligning databases of bridges and highways with the requirements of GIS-based permitting systems, which is frequently resulting in severe data management problems.
\par
According to Bazaras et al. \cite{Bazaras.2013}, safety, security, and reliability are indeed three very important aspects that have to be considered to increase the overall quality of transport processes. Therefore, apart from regular driving training and state-of-the-art equipment, detailed risk evaluation is regarded to be a key issue in terms of OHC routing. The authors underline that a fuzzy multi-criteria decision making tool is beneficial in the course of route selection, as it allows taking into account the variety of influencing factors. Apart from road quality, turning radius, road gradients, corridor widths and heights, bridge carrying capacities, reloading and storing opportunities, regular traffic intensity on route segments, and seasonal specialties, a multi-criteria system might also include an assessment of required road texture improvements and other structural adaptations or changes.
\par
In summary, the respective literature shows that taking adequate account of road gradients and bridge carrying capacities as important influencing factors for OHC route selection is highly important. However, there are only few recent contributions to this topic. In addition to that, there is only a small number of actual use cases available. Most of them originating from the United States. We assume the underlying reason being a general lack of data and insufficient data quality on road gradients, exact bridge locations, and carrying capacities.



% \begin{itemize}
%
%   \item Commercial Solutions
%
%   \item HERE
%   \url{https://www.here.com/}
%   \item PRISMA
%   \url{https://www.prisma-solutions.com/}
%
%   \item Bentley Superload Routing
%   \url{https://www.bentley.com/en/products/product-line/asset-performance/superload-routing}
%
% \end{itemize}
