% !TeX root =./main.tex
% !TeX spellcheck = en_US

The transportation of oversize and heavyweight cargoes (OHC) represents one of the most challenging processes in freight movement, frequently leading to time-consuming and costly road closures, traffic congestions and temporary manipulations of the infrastructure. This sector is gaining massive momentum due to an increase in global trade volume and a rising demand for specialized items delivery. Among cost and time-related factors, security and safety issues have top priority when undertaking OHC transportation on the road. Planning such transports is complex due to a multitude of criteria that needs to be evaluated before determining the final transport route. Aside from physical road characteristics, road turning radius and transportation corridor widths, the maximum bridge carrying capacity is a safety-relevant parameter that determines the quality of OHC transports. Historical incidents, such as the Genoa bridge collapse in 2018, are negative examples where poor maintenance along with unqualified planning and non-compliance to standards and regulations when transporting OHC over routes that include bridge segments. This study takes up the problem of selecting the optimal route for OHC transportation and proposes an integer linear model with the objective to minimize road distance while considering maximum bridge carrying capacities. We test the model to balance cost, time and safety of OHC transportation using data of the Austrian highway network.
