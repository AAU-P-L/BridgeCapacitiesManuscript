% !TeX root =./main.tex
% !TeX spellcheck = en_US

The transportation of oversize and heavyweight cargoes (OHC) represents one of the most challenging processes in freight movement, frequently leading to time-consuming and costly road closures, traffic congestions and temporary adaptations of infrastructure. In recent years, OHC transportation is gaining massive momentum due to an increasing economic development, general technological progress, and the growing prevalence of large-scale facilities such as wind mills or power plants. Apart from cost and time-related factors, security and safety issues have top priority when undertaking OHC transports on the road. Planning such transports is a complex task due to a multitude of criteria that need to be evaluated before determining the final route. In addition to physical road characteristics, such as road turning radius and transport corridor widths, the maximum bridge carrying capacity is an important safety-relevant parameter that highly influences OHC transports. Critical incidents, such as the Genoa bridge collapse in 2018, are negative examples where poor maintenance coincides with planning insufficiencies and non-compliance to standards and regulations. In this study, we take up the problem of selecting the optimal route for OHC transports under certain restrictions. In doing so, we propose an integer linear model with the objective to minimize road distance while considering maximum bridge carrying capacities. We test the model to balance cost, time, and safety of OHC transportation using data of the Austrian highway network.
